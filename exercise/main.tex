\makeatletter
\providecommand*{\input@path}{}
\edef\input@path{{../libs/smile},\input@path}
\makeatother

\documentclass{exercise}

\usepackage{fontspec}
\setsansfont[
	Ligatures=TeX,
	BoldFont={* Medium},
]{Fira Sans}
\setmonofont[
	Path = ./fonts/,
	Scale = .9,
	Extension = .ttf,
	Contextuals=Alternate,
	BoldFont={*-Bold},
	UprightFont={*-Regular},
]{Fira Code}

\def\link #1 to #2;{\def\ULdepth{.5pt}\def\ULthickness{.1pt}\uline{\href{#2}{#1}}}

\title{Functional Programming II}
\author{Lukas Pietzschmann}
\email{lukas.pietzschmann@uni-ulm.de}
\topic{Lenses}
\institute{Institute of Software Engineering and Programming Languages}
\uni{Ulm University}
\sheet{1}
\date{May~17\textsuperscript{th}, 2024}

\begin{document}
	\maketitle

	\begin{exercise}[Some hints]{}
		If you ever get stuck, don't even think about asking me for help! While that was
		obviously a joke, I'll leave you some notes on the documentation here, just in
		case you want to try helping yourself first.

		You can find the lens package's documentation \link here to
		https://hackage.haskell.org/package/lens;. If you scroll down to \link Modules
		to https://hackage.haskell.org/package/lens\#modules;, you can find an overview
		of all the modules that are part of the lens package. Here's a little summary of
		the most important ones:
		\begin{description}
			\item[Control.Lens.Lens] Here you can find the \texttt{Lens} type itself and
				some functions to work with it. We won't need most of them, but you can,
				e.g., find the \link\texttt{\&}-operators to
				https://hackage.haskell.org/package/lens-5.2.3/docs/Control-Lens-Lens.html\#g:4;
				and the \link \texttt{lens} function to
				https://hackage.haskell.org/package/lens-5.2.3/docs/Control-Lens-Lens.html\#v:lens;
				here.
			\item[Control.Lens.Operators] Here are all the operators we used. Since this
				module is only a listing of operators from other modules, you can find
				hyperlinks to the actual modules on the left or above every section. You
				can press \texttt{<CTRL-f>} on this page, hit random symbols on your
				keyboard, and you'll probably find an operator with this name.
			\item[Control.Lens.Getter] You probably expected \link\texttt{to} to
				https://hackage.haskell.org/package/lens-5.2.3/docs/Control-Lens-Getter.html\#v:to;
				and \link\texttt{like} to
				https://hackage.haskell.org/package/lens-5.2.3/docs/Control-Lens-Getter.html\#v:like;
				to be defined here, but you'll also find the \link\texttt{view} to
				https://hackage.haskell.org/package/lens-5.2.3/docs/Control-Lens-Getter.html\#v:view;
				function and it's corresponding \link operator to
				https://hackage.haskell.org/package/lens-5.2.3/docs/Control-Lens-Getter.html\#v:-94-.;
				there.
			\item[Control.Lens.Setter] Analogous to the \texttt{Getter} module, you can
				find the \link\texttt{set} to
				https://hackage.haskell.org/package/lens-5.2.3/docs/Control-Lens-Setter.html\#v:set;
				and \link\texttt{setting} to
				https://hackage.haskell.org/package/lens-5.2.3/docs/Control-Lens-Setter.html\#v:setting;
				functions here. Along with, e.g., \link\texttt{over} to
				https://hackage.haskell.org/package/lens-5.2.3/docs/Control-Lens-Setter.html\#v:over;
				and some \texttt{set}-like operators.
			\item[Control.Lens.Prism] Here, all Prism related types and functions are
				defined. You might be primarily interested in the section \link Common
				Prisms to
				https://hackage.haskell.org/package/lens-5.2.3/docs/Control-Lens-Prism.html\#g:4;.
			\item[Control.Lens.Traversal] Basically, we only used two functions from
				here; \link\texttt{traverse} to
				https://hackage.haskell.org/package/lens-5.2.3/docs/Control-Lens-Traversal.html\#v:traverse;,
				and \link\texttt{over} to
				https://hackage.haskell.org/package/lens-5.2.3/docs/Control-Lens-Traversal.html\#v:over;.
				But the \link Common Traversals to
				https://hackage.haskell.org/package/lens-5.2.3/docs/Control-Lens-Traversal.html\#g:5;
				section might contain some more interesting functions.
		\end{description}
		If you are looking for a specific function, you can also always use the Quick
		Jump button on top of the page. There, you can just type the name of the
		function and the search box will present you a link to its documentation.
	\end{exercise}

	\begin{exercise}[Exercise 0]{Setup}
		To get started, you should have the lens package installed. You can use cabal for this:
		\begin{lstlisting}[language=bash]
			cabal install --lib lens
		\end{lstlisting}
		On my system, I have GHC \texttt{9.4.8} and lens \texttt{5.2.3} installed.

		You can check if your installation works by loading the provided
		\texttt{Main.hs} file into GHCi and printing the value of
		\texttt{currentTextWidth}.
		\begin{lstlisting}[language=bash]
			ghci Main.hs
			ghci> currentTextWidth
			Just "88"
		\end{lstlisting}
		If this works, you should be good to go.
	\end{exercise}

	\begin{exercise}{Example Title}
		Example Init Task:
		\begin{tasks}
			\item Task 1
			\item Task 2
			\item Task 3
		\end{tasks}
	\end{exercise}
\end{document}