\makeatletter
\providecommand*{\input@path}{}
\edef\input@path{{../libs/smile},\input@path}
\makeatother

\documentclass{exercise}

\usepackage{fontspec}
\usepackage[sfdefault]{FiraSans}
\setmonofont[
	Path = ./fonts/,
	Scale = .9,
	Extension = .ttf,
	Contextuals=Alternate,
	BoldFont={*-Bold},
	UprightFont={*-Regular},
]{Fira Code}

\makeatletter
\usepackage{fontawesome5}
\def\ergo{\smile@lst@style@comment\raisebox{.1pt}{\scalebox{.8}{\faCaretRight}}}
\makeatother

\usepackage[verbatim]{lstfiracode}
\lstdefinestyle{firastyleb}{style=FiraCodeStyle,style=smile@lst@base}
\lstset{style=firastyleb}

\newcommand\h[2][]{\lstinline[language=haskell,#1]{#2}}
\lstnewenvironment{haskell}{\lstset{language=haskell}}{}

\def\link #1 to #2;{\def\ULdepth{.5pt}\def\ULthickness{.1pt}\uline{\href{#2}{#1}}}

\usetikzlibrary{tikzmark}
\def\m#1{\tikzmark{#1}}

\title{Functional Programming II}
\author{Lukas Pietzschmann}
\email{lukas.pietzschmann@uni-ulm.de}
\topic{Lenses}
\institute{Institute of Software Engineering and Programming Languages}
\sheet{4}
\date{May~17\textsuperscript{th}, 2024}

\begin{document}
	\maketitle

	\begin{exercise}[Some useful pointers]{}
		If you ever get stuck, you can try helping yourself first. I'll leave you some
		notes on the documentation here.

		You can find the lens package's documentation \link here to
		https://hackage.haskell.org/package/lens;. If you scroll down to \link Modules
		to https://hackage.haskell.org/package/lens\#modules;, you can find an overview
		of all the modules that are part of the package. Here's a little summary of the
		most important ones:
		\begin{description}
			\item[Control.Lens.Lens] Here you can find the \texttt{Lens} type itself and
				some functions to work with it. We won't need most of them, but you can,
				e.g., find the \link\texttt{\&}-operators to
				https://hackage.haskell.org/package/lens-5.2.3/docs/Control-Lens-Lens.html\#g:4;
				and the \link \texttt{lens} function to
				https://hackage.haskell.org/package/lens-5.2.3/docs/Control-Lens-Lens.html\#v:lens;
				here.
			\item[Control.Lens.Operators] Here are all the operators we used. Since this
				module is only a listing of operators from other modules, you can find
				hyperlinks to the actual modules on the left or above every section. You
				can press \texttt{<CTRL-f>} on this page, hit random symbols on your
				keyboard, and you'll probably find an operator with this name.
			\item[Control.Lens.Getter] You probably alredy expected \link\texttt{to} to
				https://hackage.haskell.org/package/lens-5.2.3/docs/Control-Lens-Getter.html\#v:to;
				and \link\texttt{like} to
				https://hackage.haskell.org/package/lens-5.2.3/docs/Control-Lens-Getter.html\#v:like;
				to be defined here, but you'll also find the \link\texttt{view} to
				https://hackage.haskell.org/package/lens-5.2.3/docs/Control-Lens-Getter.html\#v:view;
				function and its corresponding \link operator to
				https://hackage.haskell.org/package/lens-5.2.3/docs/Control-Lens-Getter.html\#v:-94-.;
				there.
			\item[Control.Lens.Setter] Analogous to the \texttt{Getter} module, you can
				find \link\texttt{set} to
				https://hackage.haskell.org/package/lens-5.2.3/docs/Control-Lens-Setter.html\#v:set;
				and \link\texttt{setting} to
				https://hackage.haskell.org/package/lens-5.2.3/docs/Control-Lens-Setter.html\#v:setting;
				here. Along with, e.g., \link\texttt{over} to
				https://hackage.haskell.org/package/lens-5.2.3/docs/Control-Lens-Setter.html\#v:over;
				and some \texttt{set}-like operators.
			\item[Control.Lens.Prism] Here, all prism related types and functions are
				defined. You might be primarily interested in the section \link Common
				Prisms to
				https://hackage.haskell.org/package/lens-5.2.3/docs/Control-Lens-Prism.html\#g:4;.
			\item[Control.Lens.Traversal] Basically, we only used two functions from
				here: \link\texttt{traverse} to
				https://hackage.haskell.org/package/lens-5.2.3/docs/Control-Lens-Traversal.html\#v:traverse;,
				and \link\texttt{over} to
				https://hackage.haskell.org/package/lens-5.2.3/docs/Control-Lens-Traversal.html\#v:over;.
				But the \link Common Traversals to
				https://hackage.haskell.org/package/lens-5.2.3/docs/Control-Lens-Traversal.html\#g:5;
				section might contain some more interesting functions.
		\end{description}
		If you are looking for a specific function, you can also always use the Quick
		Jump button on top of the page, or just press \texttt{s}. There, you can just
		type the name of the function and the search box will present you a link to its
		documentation.
	\end{exercise}

	\begin{exercise}[Exercise 0]{Setup}
		To get started, you should have the lens package installed. You can use cabal for this:
		\begin{lstlisting}[language=bash]
			cabal install --lib lens
		\end{lstlisting}
		On my system, I have GHC \texttt{9.4.8} and lens \texttt{5.2.3} installed, but
		older versions should also work just fine (famous last words).

		You can check if your installation works by loading the provided
		\texttt{tasks.hs} file into GHCi and evaluating \texttt{test}.
		\begin{haskell}
			ghci tasks.hs
			ghci> test
			((1,2),3,4)
		\end{haskell}
		If this prints the correct result, you should be good to go.
	\end{exercise}

	\begin{exercise}{Getting familiar with lenses}
		Let's start easy by composing some lenses and accessing and changing values.
		\begin{tasks}
			\item Combine multiple different lenses to form a new one that focuses on
				the string \h{"Hi"} inside \h{(1, ("Hi", "Ho"), 2)}.
			\item Now, use this lens to retrieve the focused part. Try both the operator
				and function for this.
			\item Change \h{"Hi"} to a integer value of your choice and bind the result
				to a name. Again, try the operator and function.
			\item Finally, use the lens on the updated tuple, to multiply the integer
				value you set in the last task with \h{11}.
			\item Now we change things up a little bit. The tuple now contains lists of
				stings, with \h{"Hi"} being in the first one: \h{(["Hi", "Ho"], ["He",
				"Hu"])}. Change your lens so it still focuses on \h{"Hi"}.
			\item Can you think of another lens that can focus on \h{"Hi"}?
			\item Finally, we no longer want to focus on just one element, but on
				\h{"Hi"} and \h{"He"} at the same time. Write a lens that does exactly
				this.
			\item Now use this lens to change both \h{"Hi"} and \h{"He"} to a string of
				your choice.
		\end{tasks}
	\end{exercise}

	\begin{exercise}{We need to go deeper}
		Now that you feel comfortable with lenses, let's go a little further. In this
		exercise, we will be working with a file system. I have prepared some ADTs in
		\texttt{tasks.hs}, go take a look at them~\ldots~Alright, let's get started.
		\begin{tasks}
			\item We will start simple by creating a lens for each element of the
				\h{Metadata} ADT.
		\end{tasks}
		In the following, we will use this as an example:
		\begin{haskell}
example :: FileSystem
example = Folder "root" [
    File $ Doc Text (Metadata ".zshenv" "root") "",
    Folder "home" [
        Folder "luke" [
            File $ Doc Text (Metadata ".zshenv" "luke") "export EDITOR=nvim",
            File $ Doc Text (Metadata ".zsh_history" "luke") "sudo dnf rm java"
          ]
      ]
  ]
		\end{haskell}
		\begin{tasks}[resume*]
			\item Without evaluating it, determine what the expression \h{example ^.
				_File . metadata} would evaluate to?
			\item If you have not figured out the previous task: it does not evaluate to
				anything, as there's a type error. Implement a fix for it, without
				changing the expression.
			\item How could we have changed the above expression to fix the error, while
				not changing what the lens focuses on?
		\end{tasks}
		Aren't lenses, prisms, and traversals nice? But for the remaining tasks of this
		exercise, we need a new thing~\ldots~\h{Fold}s! But don't worry, I will gently
		introduce you to them.

		Imagine, you want to modify the contents of a specific file, but you don't know
		its exact path. The only thing you know is its name. After the following tasks,
		you will have written functions that can focus on a specific file, no matter
		its index in the list.
		\begin{tasks}[resume*]
			\item To get into the right mood, forget lenses for a moment. Let's start by
				writing a plain old Haskell function \h{searchFiles :: String -> [Document]
				-> [Document]} using \h{filter}. For now, we assume that any input to
				this function is a list with one layer of documents:
				\begin{haskell}
searchFiles ".zshenv" [
  Doc Text (Metadata "lost+found" "root") "",
  Doc Text (Metadata ".zshenv" "luke") "export EDITOR=nvim",
  Doc Text (Metadata ".zsh_history" "luke") "sudo dnf rm java"
]
§\ergo{} Doc Text (Metadata ".zshenv" "luke") "export EDITOR=nvim"§
				\end{haskell}
		\end{tasks}
		Now, we need a way to handle arbitrary nesting depths. We can tackle this by
		flattening everything into a single list of files.
		\begin{tasks}[resume*]
			\item \m{op1}Implement \h{flattenFolders :: File Document -> [Document]}. This
				function takes a directory tree and flattens it to a list of documents.
			\item \m{op2}Now, implement \h{searchFiles' :: String -> File Document ->
				[Document]}. This function should work exacly like \h{searchFiles}, but
				for directory trees.
		\end{tasks}
		\begin{tikzpicture}[o]%
			\node[shift={(-20mm,1mm)}] at (pic cs:op1) {\color{gray}Optional};
			\node[shift={(-20mm,1mm)}] at (pic cs:op2) {\color{gray}Optional};
		\end{tikzpicture}%
		Now it's lens time! The cool thing with lenses is, that we don't need to do the
		flattening explicitly. You already know that \h{^..} gives us a flat list of all
		targets of a traversable. So providing a \h{Traversable} instance for \h{File}
		should do the trick. But a \h{Traversal} also needs to be a \h{Functor} and
		\h{Foldable}. Implementing all those instances looks like a lot of work. Luckily
		\h{^..} also works on \h{Fold}s, and in order to create a \h{Fold} from
		\h{File}, we only need a \h{Foldable} instance!
		\begin{tasks}[resume*]
			\item Implement an instance of \h{Foldable} for \h{File} that folds over all
				documents in the file hierarchy, no matter the depth.
		\end{tasks}
		Awesome! Now we can use the \link\h{folded} to
		https://hackage.haskell.org/package/lens-5.3.1/docs/Control-Lens-Combinators.html\#v:folded;
		function to build a \h{Fold} from our \h{Foldable} and use \h{^..} to view a
		flat list of all targets of the fold!
		\begin{haskell}
			example ^.. folded
		\end{haskell}
		As a last step, we now need a way to filter this list in the lens-way.
		\begin{tasks}[resume*]
			\item Look through \link Control.Lens.Fold to https://hackage.haskell.org/package/lens-5.3.1/docs/Control-Lens-Fold.html;
				and search for a function that can be composed with a \h{Fold} and be
				used for filtering.
			\item Now use this function like the \h{filter} from task \emph{e)}, so that
				we can implement \h{searchFiles''} and make it work like this:
				\begin{haskell}
example ^.. folded . searchFiles'' ".zshenv" . content
§\ergo{} ["", "export EDITOR=nvim"]§
				\end{haskell}
		\end{tasks}
		By the way, if we are sure that there will be at maximum a single file with this
		name, we can also use \h{^?} to get a \h{Maybe} instead of a list.
		\begin{haskell}
example ^? folded . searchFiles'' ".zsh_history" . metadata . author
§\ergo{} Just "Luke"§
		\end{haskell}
		\begin{tasks}[resume*]
		\item With this new superpower unlocked, implement \h{searchAuthor} in the same
			way to focus on all files from a specific author.
			\begin{haskell}
example ^.. folded . searchAuthor "luke" . metadata . title
§\ergo{} [".zshenv",".zsh\_history"]§
			\end{haskell}
		\item Last but not least, implement \h{filesWithAuthor} to search for files with
			a specific author and name.
			\begin{haskell}
example ^?! folded . filesWithAuthor ".zshenv" "luke" . content
§\ergo{} ""§
			\end{haskell}
		\end{tasks}
	\end{exercise}

	\begin{exercise}{Reinventing the wheel}
		In this exercise we want to reimplement some common library function to get a
		better understanding of how they work.

		You might remember our exploration of the \h{Lens} type from the lecture. Here's
		a little refresher:
		\begin{haskell}
			type Lens s t a b = forall f. Functor f => (a -> f b) -> s -> f t
		\end{haskell}
		Depending on the operation performed on the lenses target, Haskell can select a
		fitting functor automagically. The functor then carries out the operation.
		\begin{tasks}
		\item The set-like operator \h{.~} makes use of the \h{Identity} functor, so it
			requires a lens of type \h{(a -> Identity b) -> s -> Identity t}. Knowing
			this, reimplement this operator. We'll name it \h{.~.}\footnote{It kinda
			looks like a person with a mustache, but upside down
			\raisebox{1.2ex}{\rotatebox{180}{\emoji{disguised-face}}}} and it should
			have the following type signature:
			\begin{haskell}
				(.~.) :: ((a -> Identity b) -> s -> Indentity t) -> b -> s -> t
			\end{haskell}
			If you're a bit rusty on how \h{Identity} works and how you get things out
			of it, check out its \link documentation to
			https://hackage.haskell.org/package/base-4.19.1.0/docs/Data-Functor-Identity.html\#t:Identity;.
		\item Next, reimplement the operator for \h{over}. It also uses the \h{Identity}
			functor. Your implementation should have the following type signature:
			\begin{haskell}
				(%~.) :: ((a -> Identity b) -> s -> Identity t) -> (a -> b) -> s -> t
			\end{haskell}
		\item The last set-like operator you want to reimplement is \h{*~}. It
			multiplies the lense's target by a given value, and --- you already guessed
			it --- also used the \h{Identity} functor. Here's its signature:
			\begin{haskell}
				(*~.) :: Num a => ((a -> Identity a) -> s -> Identity t) -> a -> s -> t
			\end{haskell}
		\end{tasks}
		Set-like function are only one side of the coin. In the next task, we want to
		tackle \h{view}, or more precisely, the \h{^.} operator. Insead of the
		\h{Identity} functor, it uses \h{Const}. Again, here's a link to its \link
		documentation to
		https://hackage.haskell.org/package/base-4.19.1.0/docs/Control-Applicative.html\#t:Const;,
		so you can familiarize yourself with it. Operators using \h{Const}, take lenses
		looking like this: \h{(a -> Const a b) -> s -> Const a t}.
		\begin{tasks}[resume*]
			\item Implement the view operator \h{^.} respecting the following type signature:
				\begin{haskell}
					(.^.) :: s -> ((a -> Const a b) -> s -> Const a t) -> a
				\end{haskell}
		\end{tasks}
		The other view-like functions and operators are a bit more complex and work
		trhough way more abstractions, so we'll stop here. Hopefully, this took away the
		type magic of lenses a bit. If you still have questions, I'll be happy to have a
		little chat about it.
	\end{exercise}
\end{document}